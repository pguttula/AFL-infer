% Math mode
%-----------
\newenvironment{nop}{}{}
\newenvironment{smathpar}{
\begin{nop}\small\begin{mathpar}}{
\end{mathpar}\end{nop}\ignorespacesafterend}

% Theorem
%--------

\theoremstyle{plain}
\newtheorem{axiom}{Axiom}[section]
\newtheorem{theorem}{Theorem}[section]
\newtheorem{lemma}[theorem]{Lemma}
\newtheorem{proposition}[theorem]{Proposition}
\newtheorem{corollary}[theorem]{Corollary}
\theoremstyle{definition}
\newtheorem{definition}[theorem]{Definition}

\newenvironment{example}[1][Example]{\begin{trivlist}
\item[\hskip \labelsep {\bfseries #1}]}{\end{trivlist}}
\newenvironment{remark}[1][Remark]{\begin{trivlist}
\item[\hskip \labelsep {\bfseries #1}]}{\end{trivlist}}

% Decorations
%-----------
\newenvironment{decoration}
  {\color{blue}\begin{array}{l}}
  {\end{array}}

% New colors
%------------
\definecolor{Bittersweet}{rgb}{1.0, 0.44, 0.37}
\definecolor{MidnightBlue}{rgb}{0.0, 0.2, 0.4}
\definecolor{BrightBlue}{rgb}{0.0, 0.2, 0.7}

% Listings
%----------
\newcommand{\lstml}{
\lstset{ %
language=ML, % choose the language of the code
basicstyle=\footnotesize\ttfamily,       % the size of the fonts that are used for the code
keywordstyle=\color{Bittersweet},
% numbers=left,                   % where to put the line-numbers
numberstyle=\tiny,      % the size of the fonts that are used for the line-numbers
stepnumber=1,                   % the step between two line-numbers. If it is 1 each line will be numbered
numbersep=5pt,                  % how far the line-numbers are from the code
showspaces=false,               % show spaces adding particular underscores
showstringspaces=false,         % underline spaces within strings
showtabs=false,                 % show tabs within strings adding particular underscores
% frame=single,                   % adds a frame around the code
tabsize=2,                      % sets default tabsize to 2 spaces
captionpos=b,                   % sets the caption-position to bottom
breaklines=true,                % sets automatic line breaking
breakatwhitespace=false,        % sets if automatic breaks should only happen at whitespace
commentstyle=\itshape\color{BrightBlue},
%escapeinside={\%*}{*)},         % if you want to add a comment within your code
mathescape=true,
morekeywords={oper, txn, invariant, module, begin, match, when, @@deriving, not, : , txn_do, do, SQL/\\}
}}
\lstnewenvironment{ocaml}
    { % \centering
			\lstml
      \lstset{}%
      \csname lst@setfirstlabel\endcsname}
    { %\centering
      \csname lst@savefirstlabel\endcsname}
\newcommand{\ocamlinline}[1]{\lstinline[language=ML,
                                        basicstyle=\footnotesize\ttfamily, 
                                        keywordstyle=\color{Bittersweet},
                                        mathescape=true]!#1!}

% SQL Trace 
% ----------
\newcommand{\lstsql}{
\lstset{ %
  language=SQL, % choose the language of the code
  basicstyle=\footnotesize\ttfamily,       % the size of the fonts that are used for the code
  keywordstyle=\color{MidnightBlue},
  % numbers=left,                   % where to put the line-numbers
  numberstyle=\tiny,      % the size of the fonts that are used for the line-numbers
  stepnumber=1,                   % the step between two line-numbers. If it is 1 each line will be numbered
  numbersep=5pt,                  % how far the line-numbers are from the code
  showspaces=false,               % show spaces adding particular underscores
  showstringspaces=false,         % underline spaces within strings
  showtabs=false,                 % show tabs within strings adding particular underscores
  % frame=single,                   % adds a frame around the code
  tabsize=2,                      % sets default tabsize to 2 spaces
  captionpos=b,                   % sets the caption-position to bottom
  breaklines=true,                % sets automatic line breaking
  breakatwhitespace=false,        % sets if automatic breaks should only happen at whitespace
  commentstyle=\itshape\color{BrightBlue},
  %escapeinside={\%*}{*)},         % if you want to add a comment within your code
  mathescape=true,
  morekeywords={BEGIN, COMMIT, ROLLBACK}
}}
\lstnewenvironment{sqltrace}
    { % \centering
			\lstsql
      \lstset{}%
      \csname lst@setfirstlabel\endcsname}
    { %\centering
      \csname lst@savefirstlabel\endcsname}

\newcommand{\sql}[1]{\lstinline[language=SQL,
                                basicstyle=\footnotesize\ttfamily, 
                                keywordstyle=\color{BrightBlue},
                                breaklines=true,
                                breakatwhitespace=false,
                                mathescape=true,
                                morekeywords={BEGIN, COMMIT, ROLLBACK}]!#1!}

% ALGORITHM2E
% ------------
% code formatting
\SetAlFnt{\sffamily}
\renewcommand\ArgSty{\normalfont\sffamily}
\renewcommand\KwSty[1]{\textnormal{\textbf{\sffamily#1}}\unskip}
\SetAlCapFnt{\normalfont\sffamily\large}
\renewcommand\AlCapNameFnt{\sffamily\large}

% comment formatting
\newcommand\algcommfont[1]{\footnotesize\ttfamily\textcolor{blue}{#1}}
\SetCommentSty{algcommfont}


% Formatting
%---------
\newcommand{\C}[1]{\code{#1}}
\newcommand{\tuplee}[1]{\langle #1 \rangle}
\newcommand*{\rom}[1]{\expandafter\romannumeral #1}

% Formatting commands
% -------------------
\newcommand{\code}[1]{{\tt #1}}
\newcommand{\spc}[0]{\quad}
\newcommand{\ALT}{~\mid~}
\newcommand{\rel}[1]{{R}_{\mathit{#1}}}
\newcommand{\conj}{~\wedge~}
\newcommand{\disj}{~\vee~}
\newcommand{\txnimp}{\mbox{${\cal T}$}}
\newcommand{\ssnimp}{{\sc SsnImp}\xspace}
%\newcommand{\coloneqq}{::=}
\newcommand{\qqquad}{\quad\quad}
\newcommand{\cskip}{\C{SKIP}}
%\newcommand{\defeq}[0]{ \triangleq }
\newcommand{\op}{\textsf{op}}
\newcommand{\E}{\mathcal{E}}
\newcommand{\I}{\mathbb{I}}
\newcommand{\F}{{\sf F}}
\newcommand{\G}{{\sf G}}
\newcommand{\D}{\mathcal{D}}
\newcommand{\T}{\mathcal{T}}
\renewcommand{\P}{\mathcal{P}}
\newcommand{\B}[1]{\small\bf #1}
\newcommand{\wmax}{{\sf W}_{max}}
\newcommand{\nmax}{{\sf N}_{best}}
